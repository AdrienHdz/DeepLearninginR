\documentclass[6pt,letter,french]{article} 
\usepackage{babel}

%\usepackage[latin1]{inputenc}
\usepackage[parfill]{parskip} % Activate to begin paragraphs with an empty line rather than an indent
\usepackage{amsmath,amsthm,amssymb,bbm} %math stuff
\usepackage{placeins} % FloatBarrier
\usepackage{fancyhdr}
\usepackage{lastpage}
\usepackage{float}    % for fig.pos='H'
\usepackage{rotfloat} % for sidewaysfigure
%\usepackage{subfig}   % for subfigure
\usepackage{subcaption}  % an alternative package for sub figures
\usepackage{comment}
\usepackage[round]{natbib}   % omit 'round' option if you prefer square brackets
\bibliographystyle{plainnat}
\usepackage{setspace} %Spacing
\usepackage{graphicx,graphics}
\usepackage{booktabs,tabularx}
\usepackage{enumerate}
\usepackage{makecell}
\usepackage{xfrac}
\restylefloat{figure}
\usepackage{appendix}
\usepackage{color, colortbl, xcolor}
\usepackage{booktabs,dcolumn} % for use with texreg in R
\usepackage[pagebackref=true,bookmarks]{hyperref}
\hypersetup{
    unicode=false,          
    pdftoolbar=true,        
    pdfmenubar=true,        
    pdffitwindow=false,     % window fit to page when opened
    pdfstartview={FitH},    % fits the width of the page to the window
    pdftitle={004-Figures},    % title
    pdfauthor={SRB},     % author
    pdfsubject={Subject},   % subject of the document
    pdfcreator={SRB},   % creator of the document
    pdfproducer={SRB}, % producer of the document
    pdfkeywords={}, % list of keywords
    pdfnewwindow=true,      % links in new window
    colorlinks=true,       % false: boxed links; true: colored links
    linkcolor=black,          % color of internal links (change box color with linkbordercolor)
    citecolor=blue,        % color of links to bibliography
    filecolor=black,      % color of file links
    urlcolor=cyan           % color of external links
}
\usepackage{wrapfig}
\usepackage{todonotes}
\usepackage{ctable}


% my commands
\newcommand{\nd}{\noindent}
\newcommand{\ntodo}[2][]{\todo[#1]{\thesubsubsection{}. #2}}

% fancy header commands
\renewcommand{\headrulewidth}{0.3pt}
\renewcommand{\footrulewidth}{0.0pt}
\setlength{\textheight}{9.00in}
\setlength{\textwidth}{7.00in}
\setlength{\topmargin}{-1.1in}
\setlength{\evensidemargin}{-0.25in}
\setlength{\oddsidemargin}{-0.25in}
\renewcommand{\baselinestretch}{0.85}
\makeatletter
\makeatother
\lfoot{} \cfoot{ } \rfoot{{\small{\em Page \thepage \ of \pageref{LastPage}}}}

\usepackage{tcolorbox}

 \vspace{-8ex}
  \date{}
\usepackage{Sweave}
\begin{document}
\Sconcordance{concordance:report.tex:report.Rnw:%
1 192 1 51 0 4 1 4 0 7 1 4 0 7 1 4 0 259 1 4 0 5 1 4 0 5 1 4 0 6 1 4 0 %
22 1}

\pagestyle{plain}

\title{%
 Méthodes avancées en exploitation de donnée \\
  \large (MATH80619)}
\author{\begin{tabular}{ll}
    Estefan Apablaza-Arancibia & 11271806\\
        Adrien Hernandez & 11271806\\

    
\end{tabular}}
\maketitle

\section{Introduction}
In the first section of this paper, a literature review covers the neural networks and deep learners algorithms, focusing on different type of neural networks architecture; the purpose of adding multiple hidden layers and, ultimately, what are the challenges regarding the increase in computing time. Furthermore, in the methodology section, a list of deep learning projects are shown in order to understand some patterns and methods. Then, an exhaustive list of the R libraries allowing to build neural networks and deep learners models. Correspondingly, the advantages and disadvantages of each libraries, their capabilities as well as what you can expect when using them. To sum up, the last section will give concrete examples on how to implement the neural networks and deep learners models with these libraries, using real data.

\section{Revue de littérature}
\ntodo[inline]{expliquer les différent possibilité qu'il existe dans le monde deep learning (i.e. RNN, CNN , GNN , etc.). Si tu veux on peux expliquer les graphes aussi} 

\section{Méthodologie}
\ntodo[inline]{Dans méthodologie, il faut faire un brève description des méthodes.}

\section{Revue des ressources R}
\begin{table}[H]
\centering
\begin{tabular}{lllll}
Package      & Pro & Con & Requirement & CRAN URL   \\
tensorflow R &     &     & Anaconda installation      & Link  \\
Kera R       &     &     &       &       \\
MxNet        &     &     &       &      \\
BRNN        &     &     &       &      

\end{tabular}
\end{table}
\ntodo[inline]{Ici on pourrait faire un tableau avec tous les packages et donner les pour et les contres}
\subsection{Tensorflow}
\subsubsection{Installation}
\subsection{BRNN}



\begin{figure}[h]

\end{figure}
\section{Tutoriels}
\ntodo[inline]{Encore à voir... on pourrait créer des tutoriels pour peut-être comparer}
\newpage
\begin{appendix}

\section{Code}
\ntodo[inline]{Le code doit aller ici ou faire un autre document. Il peut y avoir des bouts de code dans le texte si cela aide à la compréhension. Le code complet sera fourni à part dans un autre document (pas de limite de pages ici)}
\end{appendix}


\end{document}

